Title:
When cartels split: roll call votes and majority factional warfare in the Mexico City Assembly 2006--09

Authors:
Eric Magar, ITAM
Mariel Ni�o Melka, University of Essex
Rafael Ch Dur�n, ITAM

Brief overview:
Procedural cartel theory expects majority party divisions in the floor to remain unobserved. Yet ideal points scaled with roll call votes in the Mexico City Assembly reveal a systematic factional split in the majority. Rules may explain the paradox.


Abstract:
We explore legislative party factionalism. Procedural cartel theory (Cox and McCubbins 2005) sees party factions as holding a veto on the majority agenda, removing divisive bills that never get voted. As consequence, the majority party should rarely, if ever, split in the floor, no matter how deeply divided. An investigation of final passage votes in the Mexico City Assembly reveals that factional warfare, for which the majority party has gained notoriety, is in fact plainly observable. Despite inter-semester variance, majority party ideal points --- estimated with a dynamic item response model (Martin and Quinn 2002) --- systematically cleaved in two quite distinct groups. We (plan to) inspect the rules governing the legislative process of the assembly in search for explanations of an anomaly that, for this case at least, calls the relevance of parties as units of analysis into question.
