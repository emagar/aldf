ALDF rules.

Dudas
declaraci�n de urgencia <-- �c�mo se controla?
proposiciones de urgente y obvia resoluci�n? <-- �Pueden incluir un dictamen/iniciativa? Parece que no
proposiciones con puntos de acuerdo <-- �c�mo se deciden? �pueden usarse para dict�menes de iniciativas? (El Punto de Acuerdo es un pronunciamiento del Poder Legislativo que produce efectos de definici�n respecto de problemas o soluciones de car�cter nacional o estatal.)
composici�n comisi�n de normatividad legislativa


Groups (grupos parlamentarios)

1. Deputies can belong to a group, to a coalition, or to neither and remain independent.

2. Groups are formed at the start of a three-year Legislature by three or more co-partisans (LO 85-I) one of whom serves as coordinator. Groups can freely replace coordinator at any time (LO 87).

3. Deputies can belong to one group only, and same-party deputies can caucus in one group only. Anyone splitting from parent group is ulteriorly considered independent (LO 85-I).

4. Three or more independent deputies can form a coalition at any time. Coalitions shall be given legislative resources only after all groups have been satisfied (LO 85-II)

5. Two parties with pre-existing groups that fuse together become a single group (LGI 13); this excludes formerly independent deputies --- ie. members of parties with less than three deputies as well as those who split from their original group (LGI 15). Deputies who split from their group cannot form a new group, but can join a coalition; independent deputies cannot serve as committee chairs (LGI 15).


Government committee (comisi�n de gobierno).

Sesiones GC no son p�blicas

1. GC elected by majority floor vote (EG 50).

2. All Group coordinators are members, their votes weighted in proportion to floor contingent size excluding non-groups. GC size is then doubled with members from the majority party; if no party has majority, then membership is proportional to the floor (LO 41) <-- OJO, membership to guarantee quorum control to maj, weithed votes only would let opposition use vanishing quorum.

3. GC 
a. elects its president; if no majority party, presidency rotates among three largest groups, one year each (LO 42).
b. proposes committee size [5-9] (LO 59), membership, and chairs (LO 44-II) to the floor (LO 63)
c. proposes ALDF's annual budget (LO 44-III)
d. replaces GC members and propose replacements to the floor (LO 44-IV) <-- OJO: MATTERS FOR MAJ PARTY ONLY
e. Acordar la celebraci�n de sesiones p�blicas y elaborar la agenda de los asuntos pol�ticos y de tr�mite que se tratar�n en �stas (LO 44-XI)

4. When assembly not in session (7 months yearly), GC receives bills and sends them to appropriate committee(s) (LO 44-XIII). No obligation to inform diputaci�n permanente (EG 51-III).

5. Any deputy (other than committee membership) is a non-voting member of the GC (LO 17-XII).


Directing board (mesa directiva).

1. DB channels floor proceedings.

2. Members = president, 4 vice-presidents, and 4 secretarios elected by floor majority of present (and secret ballot RGI 138), proportional to floor; replaced every month with no immediate reelection (LO 32). <--

3. President of DB
a. can extend sessions or declare recess or suspend them (LO 36-I).
b. prepares order in consultation with GC (LO 36-IV).
c. sends bill to max two committees for a report (committee presidents can request rectification)
d. can be removed, at any deputy's motion, by floor majority (LO 40). <--



Standing committees

1. Committees proportional to floor (LO 59).

2. Appointed ``definitely'' at start of legislature (LO 61).

3. There are 36 standing committees (LO 62)

4. Committee quorum is majority of members (RGI 131).

5. Report requires signature of majority of committee members (LO 63), all other committe decisions by majority of members present (LO 68). Minority report (voto particular) can be filed (LO 68).

6. Rules committee (Comisi�n de Normatividad Legislativa, Estudios y Pr�cticas Parlamentarias)
a. also reports all bills not exclusively in the jurisdiction of another standing committee (LO 64).
b. interprets LO and RGI's procedural rules (LO 66-III).

7. Committees must produce a report within 30 days, extendable to max. 90 by floor vote; if extension denied, 5 days will be granted for report (excitativa) by  DB President, after which committee will be discharged of the bill in favor of Rules committee -- which then has a lax time frame to report the bill (RGI 32; RGI 88 says 30 days, but fails to consider outcome if missed).

8. Committee decisions are signed by (RC 11-III, EMM no signature = veto?) and presided by (RC 12-II how much discretion in comm agenda? positive vs. negative?) committee chair. Chair schedules committee agenda (orden del d�a) (RC 19-II).

9. Report will present bill with any amendments voted in committee for floor consideration (RGI 87). Committee members opposing report can either reserve articles for particular floor vote or file minority report (also for particular vote) (RC 57).

10. Committee session discusses order as well as any urgent matter admitted by the committee (RC 24). <-- OJO: urgency

11. Committee will give precedence in order to bills including a punto de acuerdo (RC 33). <-- OJO: external messing in comm agenda; is chair forced to schedule? (Los dict�menes reca�dos a las iniciativas o proposiciones con punto de acuerdo que se presenten a discusi�n en la Comisi�n, deber�n atender preferentemente, al orden de prelaci�n en que fueron turnados por la presidencia de la Mesa Directiva.)

12. Committee meets at least once monthly, more often if chair, GC or DB so decide (RC 20).

(committee presidents can request rectification) (LO 36-VII)
committees are expected to report bill within 30 days


Order (orden del d�a)

1. Deputies shall register any bill or matter for discussion to DB on day prior to session. Only those deemed urgent by floor vote can be considered unregistered, and shall be discussed at the end of the order (RGI 93).

2. President shall confer with DB to possibly add elements to day's order prepared ``in consultation'' with GC -- how does GC impose order against DB? (RGI 95).

3. Session proceeds as stated in day's order prepared by DB and GC (RGI 98) <-- majority controls majority in joint bodies, but CG president might find it hard to prevent a roll in order...

4. Chamber shall only suspend session by GC proposal and majority vote (RGI 101, confuso).

5. All bills reported will be voted by roll call in the plenary, first in general (against the status quo), then its articles or sections, as decided by president, in particular (RGI 135).

5. Reports of ``propositions'' introduced as punto de acuerdo can only be voted en lo general, never en lo particular (RGI 116 Todo dictamen con proyecto de ley o decreto se discutir� primero en lo general y despu�s en lo particular cada uno de sus art�culos. Cuando conste de un solo art�culo ser� discutido s�lo en lo general. Los dict�menes que se refieren a puntos de acuerdo s�lo podr�n ser discutidos en lo general y no proceder� en ning�n sentido la discusi�n en lo particular, ni la reserva de sus resolutivos) <-- OJO: closed rule? can proposicion con punto de acuerdo be a bill?

6. A report will only be discussed in floor if text distributed 48 hours in advance, unless rules are suspended by floor majority (RGI 118) <-- OJO: can minority report be not distributed, hence not voted? Closed rule possibility.

7. A bill can proceed directly to the floor instead of committee if presented to GC 24 hours before and floor majority votes it urgent and of obvious resolution. It is then voted immediately with limited discussion (RGI 133).

8. An order motion can be raised at any time in the debate to enforce rules, DB president decides next step (RGI 124).

9. Debate can be stopped if floor majority (via President) votes suspense motion or considers another matter urgent (RGI 125-II and IV). Suspense motion returns bill to committee, but can only be invoked once (126).

General

1. Quorum requirement for plenary is half plus one of all members -- ie. 34 deputies (LO 28).

2. A new legislature shall only inherit bills pending with a report approved by the committees, those involving a constitutional requirement, and those accepted (``acuerdo'') by GC with floor approval (LO 90).

3. LO 92 regulates veto (standard, with auto-publication 10 days).

